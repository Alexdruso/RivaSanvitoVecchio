\documentclass[../../main.tex]{subfiles}

\begin{document}
    \subsection*{Alternative suggestion}

    Carlo lives in a country where lockdown policies are applied. The last time he shopped for groceries online, 
    he forgot to add to the cart olive oil, and now he needs it. Waiting until the next delivery would take too much time, 
    so Carlo decides to line up for the nearest store. 
    Carlo opens CLup, looks for the nearest store and selects it. 
    He then inserts the expected duration of his visit and the category of the item he wants to buy. 
    Unfortunately, CLup estimates that the olive oil department will be busy for the next hours. 
    CLup suggests an alternative store in proximity, where there is a free slot sooner. 
    Carlo accepts the suggestion and CLup queues Carlo to access the store. Carlo receives confirmation.

    \begin{table}[H]
      \centering
        \begin{tabular}{c m{.75\textwidth}}
        \hline
        \textbf{Use Case} & Alternative suggestion\\ \hline
        \textbf{Actor} & Store's customer\\ \hline
        \textbf{Entry condition} & The customer wants to access a store as soon as possible\\  \hline
        \textbf{Flow of events} & \begin{itemize}
                                    \item The customer selects the store they want to access
                                    \item The customer inserts the expected duration of the visit
                                    \item The customer inserts the categories of products they want to buy
                                    \item The customer confirms their intention
                                    \item The system checks if the store has a free slot for the customer
                                    \item The store does not have a free slot
                                    \item The system suggests to the customer another store nearby
                                    \item The customer accepts the suggestion
                                    \item The system asks the store to queue the customer
                                    \item The store confirms
                                  \end{itemize}\\ \hline
        \textbf{Exit condition} & The system shows a confirmation message to the user \\ \hline
        \textbf{Exceptions} & If the customer still wants to queue for their store of choice, the system queues the user for the first available time slot. \\ \hline
        \textbf{Special requirements} &\\ \hline
        \end{tabular}
    \end{table}

    \subsection*{Visit time and place estimation}

    Anna, a long time customer of the local store, wants to reserve a time slot to go grocery shopping the next week. 
    Anna opens CLup and selects the desired store and the desired date and time slot. 
    Not knowing yet which articles she might need, she does not fill the product categories section, 
    and she does not provide any visit duration estimation. Once Anna confirms her intentions to CLup, 
    the system predicts automatically the duration and the departments she is most likely to be in, 
    based on the collected history of previous visits. CLup then checks if Anna's visit is compatible with the 
    current schedule of the store, and, as the answer is positive, it shows a confirmation of the reservation to Anna.

    \begin{table}[H]
      \centering
        \begin{tabular}{c m{.75\textwidth}}
        \hline
        \textbf{Use Case} & Visit time and place estimation\\ \hline
        \textbf{Actor} & Store's customer\\ \hline
        \textbf{Entry condition} & The customer wants to reserve a visit to the store\\  \hline
        \textbf{Flow of events} & \begin{itemize}
                                    \item The customer selects the store they want to access
                                    \item The customer selects the time and date of the visit
                                    \item The customer confirms their intention
                                    \item The system estimates the expected duration of the visit and the visited departments
                                    \item The system checks if the store has a free slot for the customer
                                    \item The store has a free slot
                                    \item The system asks the store to queue the user
                                    \item The store confirms
                                  \end{itemize}\\ \hline
        \textbf{Exit condition} & The system shows a confirmation message to the user \\ \hline
        \textbf{Exceptions} & If the desired time slot and store are not compatible with the current scheduled queue, the system will suggest an alternative. \\ \hline
        \textbf{Special requirements} &\\ \hline
        \end{tabular}
    \end{table}

    \subsection*{Cancellation}

    Maurizio reserved a slot in the queue to access the nearby store through CLup, but, due to an unforeseen commitment, 
    he has to cancel the appointment. Maurizio opens CLup, selects the reservation and cancels it. 
    The system removes Maurizio from the queue and sends a confirmation to the customer. 
    Meanwhile, Patrizia wanted to access the store as soon as possible, but all the nearest time slots were busy, 
    and the system delayed her reservation. CLup, aware of the recently freed time slot, sends a notification to Patrizia offering her to take it over. 
    Patrizia opens CLup, accepts the offering and CLup queues her in the time slot.

    \begin{table}[H]
      \centering
        \begin{tabular}{c m{.75\textwidth}}
        \hline
        \textbf{Use Case} & Cancellation\\ \hline
        \textbf{Actor} & Store's customer\\ \hline
        \textbf{Entry condition} & A customer cancels their reservation\\  \hline
        \textbf{Flow of events} & \begin{itemize}
                                    \item CLup removes the customer from the queue
                                    \item CLup sends a confirmation to the customer
                                    \item CLup checks if there is some user wanting to access the store as soon as possible and scheduled for a later visit
                                    \item CLup sends a notification to the customer, offering to move up the visit
                                    \item The customer accepts the offer
                                    \item CLup queues the customer in the time slot
                                  \end{itemize}\\ \hline
        \textbf{Exit condition} & The system shows a confirmation message to the user \\ \hline
        \textbf{Exceptions} & If no user accepts, the slot is kept free. \\ \hline
        \textbf{Special requirements} &\\ \hline
        \end{tabular}
    \end{table}

    \subsection*{Access to the store}

    Alina is in line through CLup to access a store and her visit start time is near. 
    CLup sends a notification to Alina to remind her of the visit. 
    Alina departs from her home and approaches the store, showing the receipt on her IT device to the receipt scanner. 
    CLup checks if the customer arrived either too early or too late with respect to the assigned time slot. 
    Alina is perfectly in time and is allowed to enter the store. Finally, when the visit is coming to an end, 
    Alina's receipt is shown once again at the store cashier, who scans it. 
    CLup registers the end of the visit and sends a confirmation message to the cashier.

    \begin{table}[H]
      \centering
        \begin{tabular}{c m{.75\textwidth}}
        \hline
        \textbf{Use Case} & Access to the store\\ \hline
        \textbf{Actor} & Store's customer, store's cashier\\ \hline
        \textbf{Entry condition} & A customer's visit start time is near\\  \hline
        \textbf{Flow of events} & \begin{itemize}
                                    \item CLup sends the customer a notification to remind them of the visit
                                    \item The customer approaches the store
                                    \item The customer scans their receipt at the store entrance
                                    \item CLup checks if the current time is part of the receipt's time slot
                                    \item CLup allows the customer to enter the store
                                    \item The customer shops in the store
                                    \item The customer approaches the exit
                                    \item The customer pays and shows CLup's receipt to the store's cashier
                                    \item The store's cashier scans the receipt
                                    \item CLup registers the end of the visit
                                  \end{itemize}\\ \hline
        \textbf{Exit condition} & The system shows a confirmation message to the store's cashier \\ \hline
        \textbf{Exceptions} & If the customer scans the receipt outside their time slot, they are not allowed to enter the store. \\ \hline
        \textbf{Special requirements} &\\ \hline
        \end{tabular}
    \end{table}

    \subsection*{In presence queueing}

    Alessandro, a nurse, would like to stop at the store on the way home from work to do some urgent grocery shopping. 
    Unfortunately, Alessandro's smartphone is out of charge, and he cannot use CLup's application. 
    Alessandro stops at the store anyway, approaches the store assistant at the entrance, and asks for a receipt. 
    The store assistant accesses CLup and requests to queue a visitor. The system adds the customer to the queue and sends the line up receipt as a confirmation. 
    The store assistant prints the line up receipt and gives it to Alessandro, who waits his turn.

    \begin{table}[H]
      \centering
        \begin{tabular}{c m{.75\textwidth}}
        \hline
        \textbf{Use Case} & In presence queueing\\ \hline
        \textbf{Actor} & Store's customer, store assistant\\ \hline
        \textbf{Entry condition} & A customer asks a store assistant for a line up receipt\\  \hline
        \textbf{Flow of events} & \begin{itemize}
                                    \item The store assistant accesses CLup
                                    \item The store assistant requests CLup to queue the customer
                                    \item CLup adds the customer to the queue
                                    \item CLup sends the line up receipt to the store assistant as a confirmation
                                  \end{itemize}\\ \hline
        \textbf{Exit condition} & The store assistant prints the line up receipt \\ \hline
        \textbf{Exceptions} & \\ \hline
        \textbf{Special requirements} &\\ \hline
        \end{tabular}
    \end{table}

\end{document}
