\documentclass[../../main.tex]{subfiles}

\begin{document}

The first country to adopt CLup will probably be Italy. According to \href{http://dati-censimentopopolazione.istat.it/Index.aspx?DataSetCode=DICA_NUCLEI}{ISTAT}, in Italy, 
there are 16 648 813 family units and, according to \href{https://www.federdistribuzione.it/mappa-distributiva/}{FederDistribuzione}, there are 25 534 grocery stores. 
Assuming a uniform geographical distribution of the family units and the grocery stores and that only one person per family unit does the grocery shopping for the whole family, 
the system should be able to support 700 users per store. 
Moreover, due to the critical function of CLup during the ongoing pandemic, the system is expected to be widely adopted across the country. 
Therefore, CLup should be able to scale quickly to provide nationwide support for around 17 000 000 users. CLup at launch will start with a limited number of stores, users and resources, and the scale-up costs and required resources should grow at most linearly with the growth of the user base.
CLup should be able to maintain responsiveness even under the stated conditions of maximum capacity.

\end{document}
