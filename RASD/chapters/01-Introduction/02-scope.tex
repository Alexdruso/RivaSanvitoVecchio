\documentclass[../../main.tex]{subfiles}

\begin{document}

\subsection{Product}

CLup is a system that allows to handle access to supermarkets when the flux of
people is restricted. Handling properly such a situation is very important due
to the currently active Coronavirus pandemic: grocery shopping can't be avoided
since it is one of the most essential needs, thus access to supermarket cannot
be prevented, but it must be regulated to prevent overcrowding.

In particular, CLup allows customers who need to go grocery shopping to line-up
remotely (i.e., without being physically in a line outside the supermarket) and
suggests them the right time to go to the supermarket without having to form a
queue outside.

CLup allows customers either to request access to the supermarket as soon as
possible, or to book in advance an access to the supermarket at a given slot of
date and time. In both cases, the system aims at preventing overcrowding in each
area of the building. Access to the supermarket is granted only when using the
system so that CLup can actively monitor the number of people inside the
building.

If many people are in a queue for the same access slot, CLup gives the customers
possible alternatives about slots or supermarket that are less crowded.
Moreover, upon customer request, it can proactively inform them if there are
available slots in a given day or time range.

The main interface between CLup and the user is assumed to be an IT device with
an Internet connection. However, since not all people may have access to such
technologies, the system can be used, with limited functionalities, just through
a standard telephone line or in presence.

The system is completed by an administrational dashboard that allows store
managers to monitor the accesses to the supermarkets in real-time and to manage
the queuing parameters, such as the maximum number of people allowed in the
building at the same time.

\subsection{World and Shared phenomena}

\newcounter{wpcounter}

\subsubsection{World phenomena}

\begin{center}
  \begin{tabular}{|c| |p{12cm}|} 
    \hline
    ID & Phenomenon\\ [0.5ex] \hline\hline
    \stepcounter{wpcounter} WP\thewpcounter & The customer needs to go grocery
    shopping.\\
    \stepcounter{wpcounter} WP\thewpcounter & The customer arrives at the
    supermarket.\\
    \stepcounter{wpcounter} WP\thewpcounter & The customer leaves the
    supermarket.\\
    \stepcounter{wpcounter} WP\thewpcounter & The local authority asks the store
    manager to report how many people are inside the building.\\
    \stepcounter{wpcounter} WP\thewpcounter & The local authority asks the store
    manager to increase or decrease the maximum number of people allowed inside
    the building.\\
    \hline
  \end{tabular}
\end{center}

\newcounter{spcounter}

\subsubsection{Shared phenomena - controlled by the World}

\begin{center}
  \begin{tabular}{|c| |p{12cm}|} 
    \hline
    ID & Phenomenon\\ [0.5ex] \hline\hline
    \stepcounter{spcounter} SP\thespcounter & The customer asks the system to
    line up and enter the supermarket as soon as possible through an IT device.\\
    \stepcounter{spcounter} SP\thespcounter & The customer asks the system to
    book an entrance at the supermarket at a given date and time through an IT
    device.\\
    \stepcounter{spcounter} SP\thespcounter & The customer asks the system to
    line up and enter the supermarket as soon as possible through a standard
    telephone line.\\
    \stepcounter{spcounter} SP\thespcounter & The customer asks the system to
    book an entrance at the supermarket at a given date and time through a
    standard telephone line.\\
    \stepcounter{spcounter} SP\thespcounter & The customer asks the system to
    line up and enter the supermarket as soon as possible with an on-site
    device.\\
    \stepcounter{spcounter} SP\thespcounter & The customer asks the system to
    print the receipt of a request made by telephone with an on-site device.\\
    \stepcounter{spcounter} SP\thespcounter & The customer informs the system on
    the estimated duration of the visit to the supermarket.\\
    \stepcounter{spcounter} SP\thespcounter & The customer informs the system on
    the categories of products they intend to buy.\\
    \stepcounter{spcounter} SP\thespcounter & The customer scans the QR code
    receipt at the entrance of the supermarket.\\
    \stepcounter{spcounter} SP\thespcounter & The customer scans the QR code
    receipt at the exit of the supermarket.\\
    \stepcounter{spcounter} SP\thespcounter & The store manager queries the
    system for the number of people inside the building.\\
    \stepcounter{spcounter} SP\thespcounter & The store manager informs the
    system on the maximum number of people allowed inside the building.\\
    \hline
  \end{tabular}
\end{center}


\subsubsection{Shared phenomena - controlled by the Machine}

\begin{center}
  \begin{tabular}{|c| |p{12cm}|} 
    \hline
    ID & Phenomenon\\ [0.5ex] \hline\hline
    \stepcounter{spcounter} SP\thespcounter & The system shows the user a QR
    code as a receipt of a request performed through an IT device.\\
    \stepcounter{spcounter} SP\thespcounter & The system prints through an
    on-site device a QR code as a receipt of a request performed through a
    standard telephone line.\\
    \stepcounter{spcounter} SP\thespcounter & The system prints through an
    on-site device a QR code as a receipt of a request performed through the
    on-site device itself.\\
    \stepcounter{spcounter} SP\thespcounter & The system informs the customer
    that it's time to go to the supermarket to take advantage of the requested
    slot.\\
    \stepcounter{spcounter} SP\thespcounter & The system allows a customer to
    enter the supermarket.\\
    \stepcounter{spcounter} SP\thespcounter & The system gives the customer
    suggestions on less crowded slots or supermarkets.\\
    \stepcounter{spcounter} SP\thespcounter & The system informs the customer
    that a specific time slot in a range they chose in advance is available\\
    \hline
  \end{tabular}
\end{center}

\end{document}
