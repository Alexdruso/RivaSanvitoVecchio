\documentclass[../../main.tex]{subfiles}

\begin{document}

\subsection{Definitions}

\begin{description}
    
    \item[Dashboard] A panel usually containing instruments and controls.
    
    \item[Demographic] The statistical characteristics of human populations (such as age or income) used especially to identify markets.
    
    \item[Fallback method] A method used as reserve.

    \item[Lockdown policy] A lockdown policy is a requirement for people to stay where they are, usually due to specific risks to themselves or to others if they can move freely. 
    
    \item[Proxy] Authority given to a person to act for someone else

    \item[Push notification] A message that is "pushed" from the backend server or from the application to user interface, usually announced with sound and/or vibration of the device.

    \item[Receipt scanner] In the context of the present system, a receipt scanner is an optical device that can read customers' line up receipts.

    \item[Social distancing] In public health, social distancing, also called physical distancing,
                             is a set of non-pharmaceutical interventions or measures intended to prevent the spread of a contagious disease by maintaining 
                             a physical distance between people and reducing the number of times people come into close contact with each other.
\end{description}

\subsection{Acronyms}

\begin{description}
    
    \item[CLup] Customers Line-up
    
    \item[COVID-19] COronaVIrus Disease 2019.
    
    \item[DA] Domain Assumption
    
    \item[G] Goal.
    
    \item[IT device] Information Technology device.
    
    \item[QR code] Quick Response code.
    
    \item[RASD] Requirements Analysis and Specification Document.
    
    \item[SP] Shared Phenomenon.
    
    \item[UML] Unified Modeling Language.
    
    \item[WP] World Phenomenon.
\end{description}

\subsection{Abbreviations}

\end{document}
