\documentclass[../../main.tex]{subfiles}

\begin{document}

This document is structured in the following way:

\begin{enumerate}
    \item The first chapter is an introduction and an overview of the project, setting the context leading to its development, 
          the goals to be reached and providing a general description of its functionalities.

    \item The second chapter is a formal description of the domain model and the project through the extensive use of class diagrams and state machine diagrams. 
          Class diagrams provide a high level description of the domain entities and their relationships, while state machine diagrams focus on modeling the most important 
          entities through their state transitions. Here are also presented all the functional requirements and domain assumptions required to achieve the previously stated goals.

    \item In the third chapter non functional requirements are presented, and functional requirements are deepened thanks to the description of possible use cases through the use of natural language and thanks to sequence or activity diagrams, 
          and the design constraints are stated. 

    \item The fourth and last chapter carries a formal analysis of the model through the use of the open source Alloy language and tool, 
          including some configurations created by the tool.
\end{enumerate}


\end{document}
