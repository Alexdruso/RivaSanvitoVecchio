\documentclass[../../main.tex]{subfiles}

\begin{document}

This document is structured in the following way:

\begin{enumerate}
    \item The first chapter is an introduction and an overview of the project, setting the context leading to its development, 
          the goals to be reached and providing a general description of its functionalities.

    \item The second chapter serves as a more formal description of the project: it includes class diagrams, state machine diagrams, 
          and it gives details on the shared phenomena and domain models. Class diagrams give a big picture description on how the system should be structured,
          while state machine diagrams focus on the more relevant entities of the model. Here are also presented all the requirements and domain assumptions 
          the system in project must fulfill and take into considerations, in order to achieve the goals; they are presented each one after the goal it is relevant to.

    \item In the third chapter are presented the specific requirements, use cases described through the use of natural language and diagrams such as sequence/activity diagram, 
          and the design constraints the system must satisfy. A mockup is also shown as a general idea of how the end product should be, in terms of design and functionalities 
          offered to the end user.

    \item The fourth and last chapter is a formal analysis of the model, made through the use of the open source Alloy language and analyzer, 
          including a graphic representation of it obtained from Alloy Tool.
\end{enumerate}


\end{document}
