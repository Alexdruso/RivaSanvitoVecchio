\documentclass[../../main.tex]{subfiles}

\begin{document}

\subsection{Assumptions, dependencies and constraints}

\newcounter{dacounter}

\subsubsection{Domain assumptions}

{\rowcolors{2}{white}{lightgray}
\begin{table}[h!]
    % Centers the table
    \centering
    \begin{tabular}{| c | p{12cm} |}
    \hline
    % textbf used to achieve bold font
    \textbf{ID}                    & \textbf{Domain assumption} \\ \hline\hline

    \stepcounter{dacounter} DA\thedacounter          & Customers can access supermarkets only by using CLup system. \\ % not only by using smartphones but also fallback method

    \stepcounter{dacounter} DA\thedacounter          & The number of people who can access the store is either decided by the authorities, or by the manager, respecting the law. \\ % let's say the supermarket acts legally (comment this please)

    \stepcounter{dacounter} DA\thedacounter          & Supermarkets are equipped with a fallback method for queueing. \\
    
    \hline
    \end{tabular}
    \label{domain assumptions}
\end{table}
}

\end{document}