\documentclass[../../main.tex]{subfiles}

\begin{document}

\subsection{Assumptions, dependencies and constraints}

\newcounter{dacounter}

\subsubsection{Domain assumptions}

{\rowcolors{2}{white}{lightgray}
\begin{table}[h!]
    % Centers the table
    \centering
    \begin{tabular}{| c | p{12cm} |}
    \hline
    % textbf used to achieve bold font
    \textbf{ID}                    & \textbf{Domain assumption} \\ \hline\hline

    % \stepcounter{dacounter} DA\thedacounter          & Customers can access supermarkets only by using CLup system. \\ % not only by using smartphones but also fallback method

    \stepcounter{dacounter} DA\thedacounter          & The number of people who can access the store is either decided by the authorities, or by the manager, respecting the law. \\ % let's say the supermarket acts legally

    \stepcounter{dacounter} DA\thedacounter          & Customers won't try to bypass CLup system. \\

    \stepcounter{dacounter} DA\thedacounter          & The population is evenly distributed on the territory among store locations. \\ % check phrasing

    \stepcounter{dacounter} DA\thedacounter          & All customers who enter the supermarket check out with a human or automatic cashier. \\

    \stepcounter{dacounter} DA\thedacounter          & All customers only visit the areas of the supermarket they declared when reserving their entrance through the system. \\

    \stepcounter{dacounter} DA\thedacounter          & People will not form crowds independently from the length of the queue. \\


    \hline
    \end{tabular}
    \label{domain assumptions}
\end{table}
}

\end{document}