\documentclass[../../main.tex]{subfiles}

\begin{document}

\subsection{Definitions}

\begin{description}

    \item[Backend] The part of a computer system or application that 
    is not directly accessed by the user, typically responsible for 
    storing and manipulating data.
     
    \item[Elasticity] The degree to which a system is able to adapt to workload changes by provisioning and de-provisioning resources in an autonomic manner.

    \item[Layer] A logical structuring mechanism for the elements that make up a software solution.

    \item[Load balancer] A device that acts as a reverse proxy and 
    distributes network or application traffic across a number of servers.

    \item[Pilot project] An initial small-scale implementation that is used to 
    prove the viability of a project idea.

    \item[Tier] A physical structuring mechanism for a system infrastructure.

    \item[Web application] An application software that runs on a web server and is 
    accessed by the user through a web browser with an active internet connection.
    
\end{description}

\subsection{Acronyms}

\begin{description}

    \item[DD] Design Document.
    
    \item[CLup] Customers Line-up.
    
    \item[RASD] Requirements Analysis and Specification Document.
    
    \item[UML] Unified Modeling Language.

\end{description}

\subsection{Abbreviations}

\end{document}
