\documentclass[../../main.tex]{subfiles}

\begin{document}

During the whole implementation process we can rely on continuous integration (CI). 
This means merging all developers' working copies to a shared mainline several times a day. 
CI is intended to be used in combination with automated unit tests written through the practices of test-driven development. 
This is done by running and passing all unit tests in the developer's local environment before committing to the main repository. 
This helps avoid one developer's work-in-progress breaking another developer's copy. 

When a developer wants to merge their work on a different code base, their local environment is tested by CI. 
The repositories are merged only if tests pass. 

This continuous application of quality control aims to improve the quality of software, and to reduce the time taken to deliver it, by replacing the traditional practice of applying quality control after completing all development.

\end{document}

%Additional specifications on testing
%* Tests in CI that gate the deployment.