\documentclass[../../main.tex]{subfiles}

\begin{document}

Once the system is completed, it must be tested as a whole to verify that functional and non-functional requirements, as well as software system attributes and performance requirements are satisfied.

This phase of CLup's system testing can be divided in:


%Stress tests help you understand the upper limits of a system's capacity using a load beyond the expected maximum.
%Load test help you understand how a system behaves under an expected load.
\begin{itemize}

	\item \textbf{Load testing}: & this test allows for understanding how a system behaves under an expected load. 
	In this context, we expect that the system works without any performance issue until the system reaches its maximum capacity. 
	In particular, the goal of this test is to check whether CLup is able to maintain responsiveness even under the stated conditions of maximum capacity [see RASD 3.3].

	\item \textbf{Stress testing}: & this tests help to understand the upper limits of a system's capacity simulating a load beyond expected maximum. 
	The goal of this test is to check whether the upper limits of CLup's system are over the maximum capacity stated in RASD 3.3.

	\item \textbf{Performance testing}: & this test is performed to determine how a system performs in terms of responsiveness and stability under a particular workload. 
	The goal of performance testing in out context is to validate and verify the quality attributes of the system, such as scalability, reliability and resource usage.
	

\end{itemize} 

During the running phase, it is useful to rely on monitoring, to constantly check if the system is behaving as expected. 
Monitoring could be performed exploiting cloud platforms, since we could rely on them for hosting some services.

% TODO: staging before releasing each version to production.

\end{document}

%System test
%* Load + stress + performance testing.
%* on staging before releasing each version to production.
%* Monitoring, also exploiting the cloud platform.