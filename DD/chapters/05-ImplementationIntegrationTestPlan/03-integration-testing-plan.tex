\documentclass[../../main.tex]{subfiles}

\begin{document}

\subsection{Overview}

Immediately after the implementation of a component, it should be unit tested; if needed, a driver can be used in place of the components that are not yet fully implemented. 
As a consequence of choosing a bottom-up approach in the implementation process, we should choose a bottom-up strategy also for integration and testing as well. 
After the unit tests of a component succeeds, it is integrated in the system and integration testing is performed: every interface that the component uses is tested. Since a bottom-up approach is used, this process ends when the components at the top of the hierarchy are integrated in the system.
At that point, the system as a whole can be tested with system tests.

\subsection{Plan}

\begin{enumerate}

	\item At first, the DBMS API has to be tested, in order to check for misconfiguration and correctness of the interface.

	\item Then, the implementation goes on for StoreManager, ReceiptManager, StatisticComputationManager, and LineUpSuggestionsManager. 
	
\item Then, AccessManager, LineUpCancellationManager, and LineUpRequestManager are integrated and tested.
	\item Later, the implementation of NotificationManager and its integration with Notification Service can begin. 
	Following, the Phone Call Service API and Phone Call Service can be integrated into the system.

	\item Then, after the implementation and unit testing of AuthenticationManager and the integration of the Identity Provider service, we shall proceed with the integration testing the Verification API. 
	Following this step, Authentication API has to be tested. Since we do not implement this interface ourselves, instead relying on already existing standard interfaces, the testing process only involves the verification of the correct integration of OAuth API or SAML.

	\item After the deployment of the Web Server component, integration tests on the Web Server API and the whole Application Server API should be performed. 
	Finally, the TripTimeEstimationService and its relative Estimation API can be integrated into the system, and integration tests can be carried out. 

\end{enumerate}

\end{document}
%Testing plan
%* Implementation -> subsystem test (using driver) -> integrate -> integration test (for children)
%- DBMS integration testing
%- StoreManager
%- ReceiptManager
%- AccessManager
%- StatisticsComputationManager
%- LineUpSuggestionsManager
%- LineUpRequestManager
%- LineUpCancellationManager
%- notification servuce integration testing + NotificationManager
%- identity provider integration testing + AuthenticationManager
%- Web server tested it device + TripTimeEstimationService
%
