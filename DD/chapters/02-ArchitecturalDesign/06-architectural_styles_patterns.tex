\documentclass[../../main.tex]{subfiles}

\begin{document}

\subsection{Selected architectural styles and patterns}


    As already stated in the overview, the general architecture draws inspiration from the classical client-server architectural style, adapted in a multi-tiered cloud configuration.  
    This pattern is field-proven and offers enough space for future adjustments due to the separation of concerns fostered by the mapping of tiers into different layers.
    The presence of only web applications, thus implementing only the presentation layer, is typical of a thin client. The main advantage of such a configuration is the simplified management both for the customer, not forced to download any software, and for the developers, who can adopt a write once, run everywhere approach.

    The architecture implements a publish/subscribe pattern to handle notifications to the clients, exploiting existing notification services. 
    The publish/subscribe pattern also collapses into a queue-based communication protocol to ensure the highest decoupling between the web server and the application server.

    Finally, the architecture implements private CDNs to distribute static content with the highest efficiency and performance.

\subsection{Other design decisions}


    Other design decisions involve the use of HTTPS for communication between IT devices and the web server, as appropriate for a web application. 
    Moreover, the usage of HTTPS fosters security as requested by the RASD.

    No other specific technology is mandatory, even though it is advisable to implement the data tier as a relational database, 
    as CLup will be dealing with structured data. Moreover, relational databases are a mature technology, and commercial solutions offer distributed configurations, 
    which are well suited to the large predicted user base.

    Finally, to identify customers, each device should be assigned a random and unique identifier stored securely in the database. 

\end{document}
