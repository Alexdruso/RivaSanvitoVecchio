\documentclass[../../main.tex]{subfiles}

\begin{document}

\subsection{Selected architectural styles and patterns}


    As already stated in the overview, the general architecture draws inspiration from the classical client-server architectural style, adapted in a multi-tiered cloud configuration.  
    This pattern is field-proven and offers enough space for future adjustments due to the separation of concerns fostered by the mapping of tiers into different layers.
    The presence of only web applications, thus implementing only the presentation layer, is typical of a thin client. The main advantage of such a configuration is the simplified management both for the customer, not forced to download any software, and for the developers, who can adopt a write once, run everywhere approach and push updates to the clients very easily.

    The architecture implements a publish/subscribe pattern to handle notifications to the clients, exploiting existing notification services. 
    The publish/subscribe pattern also collapses into a queue-based communication protocol to ensure the highest decoupling between the web server and the application server.

    Finally, the architecture makes use of an external CDN to distribute static content with the highest efficiency and performance. The CDN caches the static contents that implement the client of the web app, while it acts as a reverse proxy for the API requests to the backend.

\subsection{Other design decisions}


    Other design decisions involve the use of HTTPS for communication between IT devices and the web server, as appropriate for a web application. 
    Moreover, the usage of HTTPS fosters security as requested by the RASD.

    No other specific technology is mandatory, even though it is advisable to implement the data tier as a relational database, 
    as CLup will be dealing with structured data. Moreover, relational databases are a mature technology, and commercial solutions offer distributed configurations, 
    which are well suited to the large predicted user base.

    Communication via telephone relies external services that can both receive and place outgoing telephone calls. Such services, which are assumed to communicate with the backend through a RESTful interface, should provide the following functionalities:
    \begin{itemize}
        \item place telephone calls;
        \item report to the application server when a new call is received
        \item receive from the application server arbitrary text and perform voice synthesis in an active call;
        \item report to the application server the input that the user provided in an active call, both using DTMF tone recognition and voice recognition
    \end{itemize}

    The notification service is assumed to offer the possibility to report to the application server, through a REST call, the outcome of the notification request according to the response provided by the end device.

    The trip time estimation service is assumed to offer a REST API that can be used by the IT devices to determine when it's time to show the reminder for the user to head to the store. Customers who placed their reservation using a telephone cannot be localized with enough precision, so they will be notified with a phone call that will be placed a fixed amount of time before the entrance time.

    Finally, to identify customers, each device should be assigned a random and unique identifier stored securely in the database. Store managers and store assistants are identified through
    an identity provider with either SAML or OAuth. 

\end{document}
