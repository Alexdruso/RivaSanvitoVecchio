\documentclass[../../main.tex]{subfiles}

\begin{document}

As already stated in the overview, the general architecture draws inspiration from the classical client-server architectural style, adapted in a multi-tiered cloud configuration.  
This pattern is field-proven and offers enough space for future adjustments due to the separation of concerns fostered by the mapping of tiers into different layers.
The presence of only web applications, thus implementing only the presentation layer, is typical of a thin client. The main advantage of such a configuration is the simplified management both for the customer, not forced to download any software, and for the developers, who can adopt a write once, run everywhere approach and push updates to the clients very easily.

The architecture implements a publish/subscribe pattern to handle notifications to the clients, exploiting existing notification services. 
The publish/subscribe pattern also collapses into a queue-based communication protocol to ensure the highest decoupling between the web server and the application server.

Finally, the architecture makes use of an external CDN to distribute static content with the highest efficiency and performance. The CDN caches the static contents that implement the client of the web app, while it acts as a reverse proxy for the API requests to the backend.

\end{document}
