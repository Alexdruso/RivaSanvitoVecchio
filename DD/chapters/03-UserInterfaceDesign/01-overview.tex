\documentclass[../../main.tex]{subfiles}

\begin{document}

The mockups of the graphical user interfaces to interact with CLup using an IT
device were presented in section 3.1.1 of the Requirements And Specification
Document. The graphical user interfaces will be translated into English and
Italian; the device settings will be used to determine the users' preferred
language. If the user preferred language is not available, English will be
chosen.

While store managers and store assistants only interact with the system through
IT devices, customers may interact with the system also using a standard
telephone line or in presence (through store assistants). The interactions with
the store assistants are human-to-human, so they are not described in this
document. The interactions through a telephone line, instead, are described in
the next paragraph, as a reference for the programmatic implementation that is
required.

\subsection{Customers interactions through a standard telephone line}

Interactions with the customers will be either initiated by the customers
themselves (e.g., to reserve their place in a queue) by calling CLup's telephone
number, or initiated by the system (e.g., to notify that it is time to leave to
reach the store in time for the expected entrance time) by calling the telephone
number the customers originally used to interact with the system. The customers
will be allowed to interact with the system using the language of their choice
between English and Italian. Different telephone numbers will be provided to the
users to let them choose their preferred language.

Voice recognition or DTMF tone recognition will be used to receive inputs from
the customers, while the system will use voice synthesis to reply.

Each store will be assigned a numeric identifier that can be used by the
customers to reference it.

As a receipt for their line up requests, customers will receive from the system
a numeric code that they will communicate before entering the store to the store
assistant, who will insert it in the store assistant application to print the QR
code that the customer will use to access the store.

\paragraph{Line up immediately}

\begin{verbatim}
System: Welcome to CLup. If you want to line up, press 1 or say line up.
        If you want to cancel a previous reservation, press 2 or say cancel.
Customer: Line up.
S: If you know the numeric identifier of the shop for which you want to line up,
   press 1 or say yes. If not, press 2 or say no.
C: No.
S: In what city is the shop for which you want to line up?
C: <City>.
S: What is the name of the shop for which you want to line up?
C: <Name of the shop>.
S: If you want to line up immediately, press 1 or say immediately.
   If you want to line up in the future, press 2 or say in the future.
C: Immediately.
S: Say or enter the estimated duration of your visit in minutes.
C: 30.
S: You asked to queue for <Name of the shop> in <City> for 30 minutes.
   Press 1 or say confirm to confirm, press 2 or say change to change something.
C: Confirm.
S: The first availability is at 3.15 PM.
   Say yes or press 1 to accept. Say no or press 2 to find an alternative.
C: Yes.
S: You have lined up to access <Name of the shop> in <City> at 3.15 PM for 30 minutes.
   This is the numeric code you will need to provide to a store assistant at the entrance
   of the shop to enter: 123456789. You can now hang up the call. If you want the code
   to be repeated, press 1 or say repeat.
C: *hangs up*
\end{verbatim}

\paragraph{Line up in the future}

\begin{verbatim}
System: Welcome to CLup. If you want to line up, press 1 or say line up.
        If you want to cancel a previous reservation, press 2 or say cancel.
Customer: *presses 1*
S: If you know the numeric identifier of the shop for which you want to line up,
    press 1 or say yes. If not, press 2 or say no.
C: *presses 1*
S: Enter the numeric identifier of the shop for which you want to line up, or say it
   one digit at a time.
C: *enters numeric identifier of the shop*
S: You have chosen <Name of the shop> in <City>. If you want to line up
   immediately, press 1 or say immediately. If you want to line up in the
   future, press 2 or say in the future.
C: *presses 2*
S: Say or enter the number of the month for the reservation of your visit.
C: *presses 1* *presses 1*
S: Say or enter the number of the day for the reservation of your visit.
C: *presses 2* *presses 2*
S: Say or enter the hour (in 24 hours format) for the reservation of your visit.
C: *presses 1* *presses 5*
S: Say or enter the estimated duration of your visit in minutes.
C: *presses 4* *presses 5*
S: You asked to queue for <Name of the shop> in <City> for 45 minutes
   on 22/11 starting from 3 PM.
   Press 1 or say confirm to confirm, press 2 or say change to change something.
C: *presses 1*
S: The first availability is at 3.15 PM.
    Say yes or press 1 to accept. Say no or press 2 to find an alternative.
C: *presses 2*
S: The following availability is at 3.45 PM.
    Say yes or press 1 to accept. Say no or press 2 to find an alternative.
C: *presses 1*
S: You have lined up to access <Name of the shop> in <City> at 3.45 PM for 45 minutes.
    This is the numeric code you will need to provide to a store assistant at the entrance
    of the shop to enter: 123456789. You can now hang up the call. If you want the code
   to be repeated, press 1 or say repeat.
C: *hangs up*
\end{verbatim}

\paragraph{Cancel a reservation}

\begin{verbatim}
System: Welcome to CLup. If you want to line up, press 1 or say line up.
        If you want to cancel a previous reservation, press 2 or say cancel.
Customer: Cancel.
S: Enter the numeric code of the reservation you want to cancel, or say it
one digit at a time.
C: 1 2 3 4 5 6 7 8 9
S: If you are sure you want to cancel your reservation at <Name of the shop> in <City>
   at <Time>, press 1 or say yes. Otherwise, press 2 or say no.
C: Yes.
S: You have cancelled your reservation at <Name of the shop> in <City> at <Time>.
   You can now hang up the call.
C: *hangs up*
\end{verbatim}

\paragraph{Noification from CLup}

\begin{verbatim}
System: This is a reminder for CLup that your visit at <Name of the shop> in <City>
        is planned at <Time>. Remember to be at the shop entrance on time.
        If you realize you will not be able to get to the shop, press 1 or say cancel.
        Otherwise, you can hang up the call.
Customer: *hangs up*
\end{verbatim}

\end{document}
